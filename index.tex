\documentclass[12pt, a4paper]{article}

% Acentuação
\usepackage[utf8]{inputenc}
\usepackage[portuguese]{babel}
\usepackage[T1]{fontenc}

% Configurando as margens
\usepackage[top = 2cm, bottom = 2cm, left = 3cm, right = 3cm]{geometry}
% Indentação do parágrafo
\setlength{\parindent}{2cm}
% Espaçamento entre parágrafo e texto
\setlength{\parskip}{1em}
% Espaçamento entre linhas
\renewcommand{\baselinestretch}{1.5}
% Identar o primeiro parágrafo das seções
\usepackage{indentfirst}
% Para usar símbolos como >=
\usepackage{amssymb}
% Para colocar texto entre $$ com \text{oi}
\usepackage{amsmath}
% Para usar listas com estilos específicos
\usepackage{enumerate}
% Para utilizar imagens
\usepackage{graphicx}
% Posicionamento de imagem
\usepackage{float}
% Para símbolo de permilagem com \textperthousand em modo texto
\usepackage{textcomp}
% Notação de conjuntos: $\mathbb{N, Z, Q, R, C}$
\usepackage{amssymb}

\renewcommand{\figurename}{Figura}
\renewcommand{\tablename}{Tabela}

\begin{document}

	\begin{titlepage}
	\begin{center}
		\line(1,0){300} \\
		[0.25in]
		\huge{\bfseries @ivansnpmaster} \\
		[2mm]
		\line(1,0){200} \\
		[1.5cm]
		\textsc{\large Material de estudo - Recorte de livros} \\
		[9cm]
	\end{center}
	
	\begin{flushright}
		\textsc{\large Soares, I. R.} \\
		ivansnpmaster@gmail.com \\
		\today
	\end{flushright}


\end{titlepage}

% Deixar em branco o restante da página
\clearpage
\setcounter{page}{1}

	
	\section{Álgebra linear}
		\subsection{Autovalores e autovetores}
		Seja $A=(a_{ij})\in\mathbb{M}_n(\mathbb{R})$. Dizemos que um valor $\lambda\in\mathbb{R}$ é um autovalor de $A$ se o sistema
\begin{equation} \label{eq:auto1}
\begin{pmatrix}a_{11} & \dots & a_{1n}\\ \vdots & \vdots & \vdots\\ a_{n1} & \dots & a_{nn}\\ \end{pmatrix}\cdot\begin{bmatrix}x_1\\ \vdots\\ x_n\\ \end{bmatrix}=\lambda\cdot\begin{bmatrix}x_1\\ \vdots\\ x_n\\ \end{bmatrix}
\end{equation}

tiver uma solução não nula. Para um tal autovalor $\lambda$, cada solução de \eqref{eq:auto1} será chamada de autovetor de $A$ associado a $\lambda$ (ou simplesmente, um autovetor de $A$).

\textbf{Notação}: Se $\lambda$ for um autovalor de uma matriz $A$, indicamos por $V(\lambda)$ ao conjunto de todos os autovetores de $A$ associados a $\lambda$ (isto é, o conjunto solução do sistema \eqref{eq:auto1}).

\textbf{Exemplo}: Considere uma matriz $A=\begin{pmatrix}-1 & 2\\ 2 & -4\\ \end{pmatrix}$

Queremos encontrar um valor $\lambda\in\mathbb{R}$ (se isso for possível) tal que o sistema
\begin{equation} \label{eq:autoex1}
\begin{pmatrix}-1 & 2\\ 2 & -4\\ \end{pmatrix}\cdot\begin{bmatrix}x_1\\x_2\\ \end{bmatrix}=\lambda\cdot\begin{bmatrix}x_1\\x_2\\ \end{bmatrix}
\end{equation}

tenha soluções não nulas. Observe que podemos escrever
$$\begin{bmatrix}x_1\\x_2\\ \end{bmatrix}=\begin{pmatrix}1 & 0\\0 & 1\\ \end{pmatrix}\cdot\begin{bmatrix}x_1\\x_2\\ \end{bmatrix}$$

Daí, a relação \eqref{eq:autoex1} pode ser escrita como
$$\begin{pmatrix}-1 & 2\\2 & -4\\ \end{pmatrix}\cdot\begin{bmatrix}x_1\\x_2\\ \end{bmatrix}=\lambda\cdot\begin{pmatrix}1 & 0\\0 & 1\\ \end{pmatrix}\cdot\begin{bmatrix}x_1\\x_2\\ \end{bmatrix}=\begin{pmatrix}\lambda & 0\\0 & \lambda\\ \end{pmatrix}\cdot\begin{bmatrix}x_1\\x_2\\ \end{bmatrix}$$

Com isso,
$$\begin{pmatrix}\lambda & 0\\0 & \lambda\\ \end{pmatrix}\cdot\begin{bmatrix}x_1\\x_2\\ \end{bmatrix}-\begin{pmatrix}-1 & 2\\2 & -4\\ \end{pmatrix}\cdot\begin{bmatrix}x_1\\x_2\\ \end{bmatrix}=\begin{bmatrix}0\\0\\ \end{bmatrix}$$
$$\left[\begin{pmatrix}\lambda & 0\\0 & \lambda\\ \end{pmatrix}-\begin{pmatrix}-1 & 2\\2 & -4\\ \end{pmatrix}\right]\cdot\begin{bmatrix}x_1\\x_2\\ \end{bmatrix}=\begin{bmatrix}0\\0\\ \end{bmatrix}$$
\begin{equation} \label{eq:autoex1a}
\begin{pmatrix}
\lambda+1 & -2\\-2 & \lambda+4\\ \end{pmatrix}\cdot\begin{bmatrix}x_1\\x_2\\ \end{bmatrix}=\begin{bmatrix}0\\0\\ \end{bmatrix}
\end{equation}

Nosso problema se reduz a encontrar um valor $\lambda\in\mathbb{R}$ tal que o sistema \eqref{eq:autoex1a} tenha uma solução não nula. Existe um teorema que diz que um tal sistema homogêneo tem solução não nula se e somente se o determinante de sua matriz de coeficientes for zero. Com isto, existirá um $\lambda$ como queremos se e somente se
$$0=\det\begin{pmatrix}\lambda+1 & -2\\-2 & \lambda+4\\ \end{pmatrix}=(\lambda+1)\cdot(\lambda+4)-2\cdot2=\lambda(\lambda+5)$$

isto é, quando $\lambda=0$ ou $\lambda=-5$. Esses valores serão os autovalores de $A$. Para cada autovalor deve-se achar os autovetores correspondentes. Substitui-se o respectivo valor de $\lambda$ encontrado em \eqref{eq:autoex1a} e resolve-se os sistemas correspondentes.

Para $\lambda=0$, o sistema \eqref{eq:autoex1a} será:
$$\begin{pmatrix}1 & -2\\-2 & 4\\ \end{pmatrix}\cdot\begin{bmatrix}x_1\\x_2\\ \end{bmatrix}=\begin{bmatrix}0\\0\\ \end{bmatrix}$$

ou
$$\begin{cases}x_1-2x_2=0\\-2x_1+4x_2=0\\ \end{cases}$$

Como a 2ª equação é a 1ª multiplicada por -2, a solução do sistema é a resolução da 1ª equação. Não é difícil ver que o conjunto solução desse sistema é $\{(2a,\;a):a\in\mathbb{R}\}$.

Para $\lambda=-5$, o sistema \eqref{eq:autoex1a} será:
$$\begin{pmatrix}-4 & -2\\-2 & -1\\ \end{pmatrix}\cdot\begin{bmatrix}x_1\\x_2\\ \end{bmatrix}=\begin{bmatrix}0\\0\\ \end{bmatrix}$$

ou
$$\begin{cases}-4x_1-2x_2=0\\-2x_1-x_2=0\\ \end{cases}$$

que tem conjunto solução igual a $\{(a,\;-2a):a\in\mathbb{R}\}$.
		
		\subsection{Polinômio característico}
		Um autovalor real de uma matriz $A=(a_{ij})\in\mathbb{M}_n(\mathbb{R})$, $n\geqslant1$, é um valor $\lambda\in\mathbb{R}$ tal que o sistema
\begin{equation} \label{eq:polcar1}
\begin{pmatrix}a_{11} & \dots & a_{1n}\\ \vdots & \vdots & \vdots\\a_{n1} & \dots & a_{nn}\\ \end{pmatrix}\cdot\begin{bmatrix}x_1\\ \vdots \\ x_n\\ \end{bmatrix}=\lambda\cdot\begin{bmatrix}x_1\\ \vdots \\ x_n\\ \end{bmatrix}
\end{equation}

tenha uma solução não nula. Inicialmente, observe que
$$\lambda\cdot\begin{bmatrix}x_1\\x_2\\ \vdots\\x_n\\ \end{bmatrix}=\lambda\cdot\begin{pmatrix}1 & 0 & \dots & 0\\0 & 1 & \dots & 0\\ \vdots & \vdots & & \vdots\\0 & 0 & \dots & 1\\ \end{pmatrix}\cdot\begin{bmatrix}x_1\\x_2\\ \vdots\\x_n\\ \end{bmatrix}=\begin{pmatrix}\lambda & 0 & \dots & 0\\0 & \lambda & \dots & 0\\ \vdots & \vdots & & \vdots\\0 & 0 & \dots & \lambda \end{pmatrix}\cdot\begin{bmatrix}x_1\\x_2\\ \vdots\\x_n\\ \end{bmatrix}$$

Com isso, tem-se que \eqref{eq:polcar1} é equivalente a
$$
\begin{pmatrix}\lambda & 0 & \dots & 0\\0 & \lambda & \dots & 0\\ \vdots & \vdots & & \vdots\\0 & 0 & \dots & \lambda \end{pmatrix}\cdot\begin{bmatrix}x_1\\x_2\\ \vdots\\x_n\\ \end{bmatrix}-\begin{pmatrix}a_{11} & a_{12} & \dots & a_{1n}\\a_{21} & a_{22} & \dots & a_{2n}\\ \vdots & \vdots & & \vdots\\a_{n1} & a_{n2} & \dots & a_{nn}\\ \end{pmatrix}\cdot\begin{bmatrix}x_1\\x_2\\ \vdots\\x_n\\ \end{bmatrix}=\begin{bmatrix}0\\0\\ \vdots\\0\\ \end{bmatrix}$$

ou
\begin{equation} \label{eq:polcar2}
\begin{pmatrix}\lambda-a_{11} & -a_{12} & \dots & -a_{1n}\\-a_{21} & \lambda-a_{22} & \dots & -a_{2n}\\ \vdots & \vdots & & \vdots\\-a_{n1} & -a_{n2} & \dots & \lambda-a_{nn}\\ \end{pmatrix}\cdot\begin{bmatrix}x_1\\x_2\\ \vdots\\x_n\\ \end{bmatrix}=\begin{bmatrix}0\\0\\ \vdots\\0\\ \end{bmatrix}
\end{equation}

A matriz de coeficientes do sistema \eqref{eq:polcar2} pode então ser escrita como $(\lambda\cdot I-A)$ que obviamente pertence a $\mathbb{M}_n(\mathbb{R})$. O sistema \eqref{eq:polcar2} terá uma solução não nula se e somente se $\det(\lambda\cdot I-A)=0$.

O polinômio característico de uma matriz $A\in\mathbb{M}_n(\mathbb{R})$ é o polinômio $p_a(t)=\det(t\cdot I-A)$.

O problema de encontrar os autovalores de uma matriz $A$ transforma-se em achar as possíveis raízes do polinômio $p_a(t)$. Como sabe-se que um polinômio de grau $n$ possui no máximo $n$ raízes distintas, conclui-se que uma matriz $A\in\mathbb{M}_n(\mathbb{R})$ possui no máximo $n$ autovalores distintos. Uma vez encontradas as raízes de $p_a(t)$, o cálculo dos autovetores associados a elas se reduz à solução de sistemas lineares.

\textbf{Exemplo:} Calcule os autovalores e os autovetores da matriz:
$$A=\begin{pmatrix}-1 & 3 & 0\\-3 & 5 & 0\\1 & 1 & 2\\ \end{pmatrix}$$

Primeiramente, seu polinômio característico:
$$p_a(t)=\det\begin{pmatrix}t+1 & -3 & 0\\3 & t-5 & 0\\-1 & -1 & t-2\\ \end{pmatrix}=(t+1)\cdot(t-5)\cdot(t-2)-[-9\cdot(t-2)]=$$
$$=(t-2)\cdot[(t+1)\cdot(t-5)+9]=(t-2)\cdot(t^2-5t+t-5+9)=$$
$$(t-2)\cdot(t^2-4t+4)=(t-2)\cdot{(t-2)}^2={(t-2)}^3$$

Logo, o único autovalor da matriz $A$ é 2 (pois 2 é a raíz tripla de $p_a(t)$). Agora, calcula-se os autovetores associados a 2, isto é, a solução do sistema:
$$\begin{pmatrix}3 & -3 & 0\\3 & -3 & 0\\-1 & -1 & 0\\ \end{pmatrix}\cdot\begin{bmatrix}x\\y\\z\\ \end{bmatrix}=\begin{bmatrix}0\\0\\0\\ \end{bmatrix}$$

ou
$$\begin{cases}3x-3y=0\\3x-3y=0\\-x-y=0\\ \end{cases}$$

Escalonando o sistema, tem-se:
$$\begin{cases}3x-3y=0\\-6y=0\\ \end{cases}$$

O que implica em $x=y=0$.

Observa-se que qualquer valor de $z\in\mathbb{R}$ induz uma solução do sistema do tipo $(0,\;0,\;z)=z(0,\;0,\;1)$. Logo, os autovetores associados a 2 são os múltiplos de $(0,\;0,\;1)$.

\textbf{Exemplo:} Considere uma matriz quase idêntica à do exemplo anterior (a diferença está no valor $a_{31}$).
$$A=\begin{pmatrix}-1 & 3 & 0\\-3 & 5 & 0\\-1 & 1 & 2\\ \end{pmatrix}$$

Seu polinômio característico é (também) $p_a(t)={(t-2)}^3$. Já que na multiplicação nas diagonais onde está o $a_{31}$ existe o número zero. Logo, 2 é o único autovalor de $A$. Para o cálculo dos autovetores, tem-se:
$$\begin{pmatrix}3 & -3 & 0\\3 & -3 & 0\\1 & -1 & 0\\ \end{pmatrix}\cdot\begin{bmatrix}x\\y\\z\\ \end{bmatrix}=\begin{bmatrix}0\\0\\0\\ \end{bmatrix}$$

ou
$$\begin{cases}3x-3y=0\\3x-3y=0\\x-y=0\\ \end{cases}$$

É fácil ver que o conjunto solução do sistema será o conjunto solução da equação $x-y=0$, o que implica que $x=y$. De novo, o valor de $z$ pode ser qualquer e portanto, tem-se que os autovetores de $A$ são vetores do tipo $(x,\;x,\;z)=x(1,\;1,\;0)+z(0,\;0,\;1)$, com $x, z\in\mathbb{R}$.
	
	\section{Cálculo diferencial e integral}
		\subsection{Teorema do Valor Médio}
		Em um intervalo aberto $(a, b)$ em uma função contínua $f$ existe um número $c$ onde $f(b)-f(a)=f'(c) \cdot (b-a)$. A reta tangente em $c$ é paralela à reta que liga os pontos $f(b)$ e $f(a)$.	
	
		\subsection{Regra de l'Hôspital}
		Quando o limite de um quociente resultar em uma indeterminação tipo $0/0$ ou $\infty/\infty$, a Regra de l'Hôspital diz que o limite dessa indeterminação é também o limite de suas derivadas: $$\lim_{x \rightarrow a} \frac{f(x)}{g(x)} = \lim_{x \rightarrow a} \frac{f'(x)}{g'(x)}$$

Essa ideia se baseia em dar um \textit{zoom} no ponto $a$, de forma que as curvas de $f$ e $g$ pareçam cada vez mais como retas, indicando a veracidade do limite do quociente das funções ser igual ao limite do quociente de suas derivadas.

		\subsection{Produtos indeterminados}
		Quando $f(x) \rightarrow 0$ e $g(x) \rightarrow \infty$ (ou $-\infty$) quando $x \rightarrow a$, o limite do produto $f \cdot g$ é considerado indeterminado do tipo $0 \cdot \infty$.
	
Há uma disputa entre as funções e precisamos saber quem ganhará. Se $f$ ganhar, o limite é $0$, se $g$ ganhar, o limite é $\infty$ (ou $-\infty$). Podendo ainda haver um equilíbrio e o limite tender a um número específico.

Podemos resolver reescrevendo $$\lim_{x \to a} f \cdot g$$ como $$\lim_{x \to a} \frac{f}{1/g} \text{ ou } \lim_{x \to a} \frac{g}{1/f}$$

Isso converte o produto indeterminado em outro do tipo $0/0$ ou $\infty/\infty$, fazendo com que seja possível aplicar a Regra de l'Hôspital.
		
		\subsection{Diferenças indeterminadas}
		Quando $f(x) \rightarrow \infty$ e $g(x) \rightarrow \infty$ quando $x \rightarrow a$, o limite da diferença $f-g$ é considerado indeterminado do tipo $\infty - \infty$.

Há uma disputa entre as funções e esse tipo de indeterminação geralmente acontece em quocientes. O primeiro passo para resolução do limite é encontrar um denominador em comum e realizar simplificações para posterior tentativa de utilização da Regra de l'Hôspital.
	
		\subsection{Potências indeterminadas}
		Existem três casos possíveis de indeterminação em limites de funcões potência, do tipo $$\lim_{x \rightarrow a} [f(x)]^{g(x)}$$

\begin{enumerate}
	\item $f(x) \rightarrow 0$ e $g(x) \rightarrow 0$, caso $0^0$
	\item $f(x) \rightarrow \infty$ e $g(x) \rightarrow 0$, caso $\infty^0$
	\item $f(x) \rightarrow 1$ e $g(x) \rightarrow \infty$, caso $1^\infty$
\end{enumerate}

Sua resolução pode ser feita de duas formas. Pode-se tomar o logarítmo natural em ambos os lados de $y=[f(x)]^{g(x)}$, deixando a função na forma $\ln y = g(x) \cdot \ln f(x)$. Isso causa uma mudança no limite, deixado-o na forma indeterminada do tipo $0 \cdot \infty$. A partir daí, pode-se deixar a função como um quociente (produto indeterminado) para posterior aplicação da Regra de l'Hôspital ou podemos reescrever a função $[f(x)]^{g(x)}$ como $e^{g(x)\cdot \ln f(x)}$, já que $x=e^{\ln x}$ e calculamos o limite dessa nova função, sendo: $$\lim_{x \rightarrow a}[f(x)]^{g(x)} = \lim_{x \rightarrow a} e^{g(x) \cdot \ln f(x)} = e^{\lim_{x \rightarrow a} g(x) \cdot \ln f(x)}$$
		\begin{enumerate}
	\item Calcular $\lim_{x\rightarrow0^{+}}x^x$
		$$\lim_{x\rightarrow0^+}x^x=\lim_{x\rightarrow0^{+}}(e^{\ln x})^x=e^{\lim_{x\rightarrow0^+}x\cdot\ln x}=e^0=1$$
	\item Calcular $\lim_{x\rightarrow0^{+}}[1+\sin(4x)]^{\cot x}$
		$$\lim_{x\rightarrow0^+}[1+\sin(4x)]^{\cot x}=\lim_{x\rightarrow0^+}\ln y=\lim_{x\rightarrow0^+}\cot x\cdot\ln [1+\sin(4x)]=$$
		$$\lim_{x\rightarrow0^+}\frac{\ln [1+\sin(4x)]}{\tan x}=\lim_{x\rightarrow0^+}\frac{\displaystyle\frac{4\cdot\cos(4x)}{1+\sin(4x)}}{\sec^2 x}=4$$
		$$\lim_{x\rightarrow0^+}y=\lim_{x\rightarrow 0^+}e^{\ln y}=e^4$$
\end{enumerate}
	
		\subsection{Esboço de curvas}
		\begin{enumerate}[a.]
	
	\item \textbf{Domínio}:
	
	É interessante saber onde a função está definida para sabermos quais intervalos analisar nas etapas seguintes.
	
	\item \textbf{Simetria}:
	
	Quando $f(x)=f(-x)$, a função é par, ou seja, possui simetria em relação ao eixo $y$.
	Quando $f(x)=-f(x)$, a função é ímpar, ou seja, possui simetria rotacionando a função em $180^{\circ}$ em torno da origem.
	É interessante saber se uma função possui simetria, pois o trabalho em descobrir o comportamento da função é menor.
	
	\item \textbf{Assíntotas}:
	
	Para verificar se a função possui assíntotas horizontais, devemos checar como ela se comporta quando tende aos infinitos. Em outras palavras, se $$\lim_{x \rightarrow \pm \infty} f(x) = L$$
	Para verificar se a função possui assíntotas verticais, é interessante verificar pontos próximos do local onde a função não está definida (extremos do domínio). Em outras palavras, se $$\lim_{x \rightarrow a} f(x) = \infty \text{ ou} -\infty$$
	Uma função pode ter uma assíntota vertical quando se aproxima de um número $a$ pela esquerda ou pela direita (ou ambos).
	
	\item \textbf{Intervalos de crescimento/caimento e números críticos}:
	
	Os intervalos de crescimento e caimento de uma função $f(x)$ são obtidos quando $f'(x) = 0$. Quando isolamos a variável independente, encontramos onde a inclinação é $0$ e podemos dividir o domínio de $f$ com base nos nesses pontos. Ao fazer isso, podemos testar pontos contidos nesses intervalos em $f'$ para verificar seu crescimento/caimento com base no sinal de $f'$.
	
	Os pontos onde $f'(x) = 0$ são números críticos $(c)$, indicando possíveis máximos ou mínimos locais.
	
	O ponto $c$ será um máximo local quando $f'(x) \textgreater 0$ em $x \rightarrow c^-$ e quando $f'(x) \textless 0$ em $x \rightarrow c^+$. De forma análoga, $c$ será um mínimo local quando $f'(x) \textless 0$ em $x \rightarrow c^-$ e quando $f'(x) \textgreater 0$ em $x \rightarrow c^+$.
	
	Para sabermos qual número $c_n$ nos possíveis $n$ pontos críticos de $f$ é um máximo ou mínimo absoluto, devemos encontrar $f(c_n)$ e verificar qual número $c$ resultou num maior número $y$ (máximo absoluto) e menor $y$ (menor absoluto).
	
	\item \textbf{Concavidade e pontos de inflexão}:
	
	A concavidade de uma função $f(x)$ é definida pelo sinal de $f''(x)$. Os pontos onde $f''(x) = 0$ são chamados de pontos de inflexão e são neles onde ocorre a mudança de concavidade de $f$. Encontrar a variável independente quando $f''(x) = 0$ divide o domínio de $f$ nos pontos onde ocorre a inflexão da curva.

\end{enumerate}
		
		\subsection{Derivação implícita}
		Consiste na derivação de ambos os lados da equação em relação a $x$ e, então, na resolução da equação isolando $y'$ ou $\displaystyle\frac{dy}{dx}$.
		\begin{enumerate}
	\item Calcular $\displaystyle\frac{dy}{dx}$ para $x^2+y^2=25$
		$$\frac{d}{dx}(x^2+y^2)=\frac{d}{dx}25$$
		$$2x+2y\frac{dy}{dx}=0$$
		$$\frac{dy}{dx}=-\frac{x}{y}$$
	\item Encontre uma equação da tangente ao círculo $x^2+y^2=25$ no ponto (3, 4)
		
		No ponto (3, 4), tem-se $x=3$ e $y=4$:
		$$\frac{dy}{dx}=-\frac{3}{4}$$
		$$y-y_0=m(x-x_0)$$
		$$y-4=-\frac{3}{4}(x-3)$$
		Ou $$3x+4y=25$$
	\item Calcular $y'$ para $x^3+y^3=6xy$ (Fólio de Descartes)
		$$x^3+y^3=6xy$$
		$$3x^2+3y^2y'=6(xy'+y\cdot1)$$
		$$x^2+y^2y'=2xy'+2y$$
		$$y^2y'-2xy'=2y-x^2$$
		$$y'(y^2-2x)=2y-x^2$$
		$$y'=\frac{2y-x^2}{y^2-2x}$$
	\item Calcular $y'$ para $\sin(y+x)=y^2\cos x$
		$$\sin(y+x)=y^2\cos x$$
		$$\cos(y+x)(y'+1)=y^2(-\sin x)+\cos x\cdot2yy'$$
		Agrupando termos semelhantes
		$$\cos(y+x)+y^2\sin x=\cos x\cdot(2yy')-\cos(y+x)y'$$
		$$\cos(y+x)+y^2\sin x=y'(\cos x\cdot2y-\cos(y+x))$$
		$$y'=\frac{\cos(y+x)+y^2\sin x}{\cos x\cdot2y-\cos(y+x)}$$
	\item Calcular $y''$ para $x^4+y^4=16$
		$$x^4+y^4=16$$
		$$4x^3+4y^3y'=0$$
		$$y'=-\frac{4x^3}{4y^3}=-\frac{x^3}{y^3}$$
		Derivando $y'$ pela Regra do Quociente:
		$$y''=\frac{d}{dx}\left(-\frac{x^3}{y^3}\right)=-\left(\frac{y^3\cdot3x^2-x^3\cdot3y^2y'}{(y^3)^2}\right)$$
		Substituindo $y'$ na equação acima:
		$$-\left(\frac{y^3\cdot3x^2-x^3\cdot3y^2\left[-\displaystyle\frac{x^3}{y^3}\right]}{y^6}\right)$$
		$$-\frac{3(x^2y^4+x^6)}{y^7}=-\frac{3x^2(y^4+x^4)}{y^7}$$
		
		Podendo ainda ser simplificada, pois $x^4+y^4=16$
		$$-\frac{3x^2(16)}{y^7}=-\frac{48x^2}{y^7}$$
\end{enumerate}
		
		\subsection{Quadratura de Gauss-Legendre}
		É uma forma de integração numérica. Evalua-se uma área sob uma linha reta pela junção de \textbf{quaisquer} dois pontos numa curva ao invés de simplesmente escolher pontos finais. A chave é escolher uma linha que \textbf{balanceia} os erros positivos e negativos.
$$I\approx c_0 f(x_0)+c_1 f(x_1)+\dots + c_{n} f(x_{n})$$

Ou,

$$I\approx \sum c_i f(x_i)$$

Onde $c_i$ (pesos) e $x_i$ devem ser determinados.

Para a Quadratura de Gauss-Legendre de dois pontos, tem-se:

$$I\approx c_0 f(x_0)+c_1 f(x_1)$$

Considerações:
\begin{itemize}
	\item São necessárias quatro equações para determinar $c_0$, $c_1$, $x_0$ e $x_1$;
	\item Assume-se que as integrais de $y=c$, $y=x$, $y=x^2$ e $y=x^3$ são computadas \textbf{exatamente} pela integral acima;
	\item Simplifica-se assumindo uma integração no intervalo $[-1, 1]$, i.e., move-se o intervalo de integração $[a,b]$ para $[-1, 1]$.
\end{itemize}
		
	\section{Resistência dos materiais}
		\subsection{Tensão e deformação - carregamento axial}
		Considerando uma barra de comprimento $L$ e seção transversal uniforme, e chamando-se de $\delta$ sua deformação sob uma carga axial $P$:

// Inserir imagem

Define-se a \textit{deformação específica normal} $\varepsilon$ da barra como sendo a \textit{deformação por unidade de comprimento}:
\begin{equation}\label{eq:def-espec}
	\varepsilon=\frac{\delta}{L}
\end{equation}

Em barra de seção transversal variável, a deformação específica normal é definida:
\begin{equation}
	\varepsilon=\lim_{\Delta x\rightarrow0}\frac{\Delta\delta}{\Delta x}=\frac{d\delta}{dx}
\end{equation}

// Inserir imagem

A porção inicial do diagrama tensão-deformação é uma linha reta. Isto significa que para pequenas deformações a tensão é diretamente proporcional à deformação:
\begin{equation}\label{eq:leidehooke}
	\sigma=E\varepsilon
\end{equation}

Esta relação é conhecida como Lei de Hooke e o coeficiente $E$ como \textit{módulo de elasticidade longitudinal} do material. A maior tensão para a qual a Equação~\eqref{eq:leidehooke} se aplica é a \textit{tensão de proporcionalidade} do material.
		
		\subsection{Deformações de barras sujeitas a cargas axiais}		
		Tratando-se de deformação elástica, se uma barra de comprimento $L$ e seção transversal uniforme de área $A$ é submetida a uma carga $P$, axial e centrada em sua extremidade, a correnpondente deformação é, a partir da junção de que $\sigma=F/A$ e das Equações~\eqref{eq:def-espec} e~\eqref{eq:leidehooke}:
$$\varepsilon=\frac{\delta}{L}$$
$$\sigma=E\varepsilon$$

Sua união,
$$\varepsilon=\frac{\sigma}{E}=\frac{P}{AE}$$

Portanto,
\begin{equation}
	\delta=\frac{PL}{AE}
\end{equation}

Se a barra for carregada em vários pontos ou consiste em várias partes com seções transversais diferentes, e ainda, possivelmente, de diferentes materiais, a deformação $\delta$ da barra deve ser expressa como o somatório das deformações nas várias partes:
\begin{equation}
	\delta=\sum_i\frac{P_iL_i}{A_iE_i}
\end{equation}
		
		\subsection{Problemas envolvendo variação de temperatura}
		Tomando uma barra $AB$, homogênea e de seção transversal uniforme, apoiada em uma superfície lisa horizontal. Se for aumentada a temperatura da barra em um valor $\Delta T$, nota-se que ela se alonga de um valor $\delta_T$ que é proporcional tanto à variação de temperatura quanto ao comprimento da barra $L$. Tem-se, então:
\begin{equation}\label{eq:def-total}
	\delta_T=\alpha(\Delta T)L
\end{equation}

Onde $\alpha$ é a constante característica do material, chamada de \textit{coeficiente de dilatação térmica}. Como $L$ e $\delta_T$ são expressos em unidades de comprimento, $\alpha$ representa uma quantidade \textit{por grau C} ou \textit{por grau F}, dependendo de como a temperatura é expressada.

À deformação total $\delta_T$ está relacionada uma deformação específica $\varepsilon_T=\delta_T/L$. Reescrevendo a Equação~\eqref{eq:def-total}:
\begin{equation}
	\varepsilon_T=\alpha(\Delta T)
\end{equation}

Onde $\varepsilon_T$ é chamada de \textit{deformação térmica específica}, uma vez que é causada por variação de temperatura na barra. No caso considerado não há tensões relacionadas com a deformação $\varepsilon_T$.

\begin{figure}[H]
	\begin{center}
	\caption{Aumento de comprimento devido o acréscimo de temperatura.}
    	\includegraphics[width=0.5\textwidth]{Resistencia-dos-materiais/Imagens/Variacao-de-temperatura.jpg}
	\end{center}
\end{figure}

Agora, um caso específico. Considerando uma barra $AB$ de comprimento $L$, colocada entre dois anteparos fixos, separados por uma distância $L$. Elevando-se $\Delta T$, o alongamento da barra é nulo, pois os anteparos impedem qualquer deformação. Sendo a barra homogênea e de seção uniforme, a deformação específica em qualquer ponto é $\varepsilon=\delta_t/L$, também nula. Entretanto, para evitar o alongamento da barra, os anteparos vão aplicar sobre ela as forças $P$ e $P'$ após a elevação da temperatura. É criado um estado de tensão na barra (sem que ocorram deformações específicas).

\begin{figure}[H]
	\begin{center}
	\caption{Aumento de temperatura sem deformação.}
    	\includegraphics[width=0.5\textwidth]{Resistencia-dos-materiais/Imagens/Variacao-de-temperatura-sem-deformacao.jpg}
	\end{center}
\end{figure}	
		
		\subsection{Estado plano de tensão}
		Adotando-se que o ponto $Q$ está submetido a um estado plano de tensões (com $\sigma_z=\tau_{zx}=\tau_{zy}=0$), que é representado pelas componentes de tensão $\sigma_x$, $\sigma_y$ e $\tau_{xy}$ relativas ao cubo elementar a seguir:

Procura-se agora determinar as componentes de tensão $\sigma_{x'}$, $\sigma_{y'}$ e $\tau_{x'y'}$, referentes ao cubo elementar que foi rodado de um ângulo $\theta$ em torno do eixo $z$, como a seguir:

expressando essas componentes em função de $\sigma_x$, $\sigma_y$, $\tau_{xy}$ e $\theta$.

Para determinar a tensão normal $\sigma_{x'}$ e a tensão de cisalhamento $\tau_{x'y'}$ que atuam na face perpendicular ao eixo $x'$, considera-se o prisma elementar de faces perpendiculares aos eixos $x$, $y$ e $x'$, como a seguir:

Chamando de $\Delta A$ a área da face inclinada, calcula-se as áreas das faces vertical e horizontal por ($\Delta A\cos\theta$) e ($\Delta A\sin\theta$), respectivamente. Com isso, as forças elementares que atuam nessas faces são as seguintes:

Não ocorrem forças atuando nas faces triangulares do prisma elementar, pois foi adotado que as componentes de tensões nessas faces são nulas.

Fazendo-se as equações de equilíbrio dessas forças em relação aos eixos $x'$ e $y'$, tem-se:
$$\sum F_{x'}=0$$
$$\sigma_{x'}\Delta A-\sigma_x(\Delta A\cos\theta)\cos\theta-\tau_{xy}(\Delta A\cos\theta)\sin\theta-\sigma_y(\Delta A\sin\theta)\sin\theta-\tau_{xy}(\Delta A\sin\theta)\cos\theta=0$$

Resolvendo para $\sigma_{x'}$:
$$\sigma_{x'}=\sigma_x\cos^2\theta+\sigma_y\sin^2\theta+2\tau_{xy}\sin\theta\cos\theta$$
		
\end{document}