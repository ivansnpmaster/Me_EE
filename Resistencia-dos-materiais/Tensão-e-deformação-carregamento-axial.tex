Considerando uma barra de comprimento $L$ e seção transversal uniforme, e chamando-se de $\delta$ sua deformação sob uma carga axial $P$:

// Inserir imagem

Define-se a \textit{deformação específica normal} $\varepsilon$ da barra como sendo a \textit{deformação por unidade de comprimento}:
\begin{equation}\label{eq:def-espec}
	\varepsilon=\frac{\delta}{L}
\end{equation}

Em barra de seção transversal variável, a deformação específica normal é definida:
\begin{equation}
	\varepsilon=\lim_{\Delta x\rightarrow0}\frac{\Delta\delta}{\Delta x}=\frac{d\delta}{dx}
\end{equation}

// Inserir imagem

A porção inicial do diagrama tensão-deformação é uma linha reta. Isto significa que para pequenas deformações a tensão é diretamente proporcional à deformação:
\begin{equation}\label{eq:leidehooke}
	\sigma=E\varepsilon
\end{equation}

Esta relação é conhecida como Lei de Hooke e o coeficiente $E$ como \textit{módulo de elasticidade longitudinal} do material. A maior tensão para a qual a Equação~\eqref{eq:leidehooke} se aplica é a \textit{tensão de proporcionalidade} do material.