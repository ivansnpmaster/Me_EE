Adotando-se que o ponto $Q$ está submetido a um estado plano de tensões (com $\sigma_z=\uptau_{zx}=\uptau_{zy}=0$), que é representado pelas componentes de tensão $\sigma_x$, $\sigma_y$ e $\uptau_{xy}$ relativas ao cubo elementar a seguir:

Procura-se agora determinar as componentes de tensão $\sigma_{x'}$, $\sigma_{y'}$ e $\uptau_{x'y'}$, referentes ao cubo elementar que foi rodado de um ângulo $\theta$ em torno do eixo $z$, como a seguir:

expressando essas componentes em função de $\sigma_x$, $\sigma_y$, $\uptau_{xy}$ e $\theta$.

Para determinar a tensão normal $\sigma_{x'}$ e a tensão de cisalhamento $\uptau_{x'y'}$ que atuam na face perpendicular ao eixo $x'$, considera-se o prisma elementar de faces perpendiculares aos eixos $x$, $y$ e $x'$, como a seguir:

Chamando de $\Delta A$ a área da face inclinada, calcula-se as áreas das faces vertical e horizontal por ($\Delta A\cos\theta$) e ($\Delta A\sin\theta$), respectivamente. Com isso, as forças elementares que atuam nessas faces são as seguintes:

Não ocorrem forças atuando nas faces triangulares do prisma elementar, pois foi adotado que as componentes de tensões nessas faces são nulas.

Fazendo-se as equações de equilíbrio dessas forças em relação aos eixos $x'$ e $y'$, tem-se:
$$\sum F_{x'}=0$$
$$\sigma_{x'}\Delta A-\sigma_x(\Delta A\cos\theta)\cos\theta-\uptau_{xy}(\Delta A\cos\theta)\sin\theta-\sigma_y(\Delta A\sin\theta)\sin\theta-\uptau_{xy}(\Delta A\sin\theta)\cos\theta=0$$

Resolvendo para $\sigma_{x'}$:
$$\sigma_{x'}=\sigma_x\cos^2\theta+\sigma_y\sin^2\theta+2\uptau_{xy}\sin\theta\cos\theta$$