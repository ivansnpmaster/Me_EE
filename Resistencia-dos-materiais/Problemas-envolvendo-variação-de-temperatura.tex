Tomando uma barra $AB$, homogênea e de seção transversal uniforme, apoiada em uma superfície lisa horizontal. Se for aumentada a temperatura da barra em um valor $\Delta T$, nota-se que ela se alonga de um valor $\delta_T$ que é proporcional tanto à variação de temperatura quanto ao comprimento da barra $L$. Tem-se, então:
\begin{equation}\label{eq:def-total}
	\delta_T=\alpha(\Delta T)L
\end{equation}

Onde $\alpha$ é a constante característica do material, chamada de \textit{coeficiente de dilatação térmica}. Como $L$ e $\delta_T$ são expressos em unidades de comprimento, $\alpha$ representa uma quantidade \textit{por grau C} ou \textit{por grau F}, dependendo de como a temperatura é expressada.

À deformação total $\delta_T$ está relacionada uma deformação específica $\varepsilon_T=\delta_T/L$. Reescrevendo a Equação~\eqref{eq:def-total}:
\begin{equation}
	\varepsilon_T=\alpha(\Delta T)
\end{equation}

Onde $\varepsilon_T$ é chamada de \textit{deformação térmica específica}, uma vez que é causada por variação de temperatura na barra. No caso considerado não há tensões relacionadas com a deformação $\varepsilon_T$.

// Inserir imagem