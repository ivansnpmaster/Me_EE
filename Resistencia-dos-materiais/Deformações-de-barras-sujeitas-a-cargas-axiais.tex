Tratando-se de deformação elástica, se uma barra de comprimento $L$ e seção transversal uniforme de área $A$ é submetida a uma carga $P$, axial e centrada em sua extremidade, a correnpondente deformação é, a partir da junção de que $\sigma=F/A$ e das Equações~\eqref{eq:def-espec} e~\eqref{eq:leidehooke}:
$$\varepsilon=\frac{\delta}{L}$$
$$\sigma=E\varepsilon$$

Sua união,
$$\varepsilon=\frac{\sigma}{E}=\frac{P}{AE}$$

Portanto,
\begin{equation}\label{eq:def-elastica}
	\delta=\frac{PL}{AE}
\end{equation}

Se a barra for carregada em vários pontos ou consiste em várias partes com seções transversais diferentes, e ainda, possivelmente, de diferentes materiais, a deformação $\delta$ da barra deve ser expressa como o somatório das deformações nas várias partes:
\begin{equation}
	\delta=\sum_i\frac{P_iL_i}{A_iE_i}
\end{equation}