\documentclass[12pt, a4paper]{article}

\usepackage[utf8]{inputenc}
\usepackage[portuguese]{babel}
\usepackage[T1]{fontenc}

\usepackage{tikz}
%\usetikzlibrary{angles,quotes} % for pic
\usetikzlibrary{decorations.pathreplacing}

% Configurando as margens
\usepackage[top = 3cm, bottom = 2cm, left = 3cm, right = 2cm, includefoot]{geometry}
% Posicionamento de imagem
\usepackage{float}
% Identar o primeiro parágrafo das seções
\usepackage{indentfirst}
% Para colocar texto entre $$ com \text{oi}
\usepackage{amsmath}
% Para usar os símbolos de conjunto
\usepackage{amssymb}
% Espaçamento entre linhas
%\renewcommand{\baselinestretch}{1.5}

\newcommand{\rot}{\text{Rot}}
\newcommand{\vt}[1]{\Vec{\textbf{#1}}}
\newcommand{\base}[1]{\hat{\textbf{#1}}}

% símbolo dos reais
\newcommand{\R}{\mathbb{R}}

% Cabeça sólida de seta - tikz
\tikzset{>=latex}

\begin{document}
    \title{Prova vetorial de $\sin(\alpha+\beta)$ e $\cos(\alpha+\beta)$}
\author{Ivan Ribeiro}
\date{\today}
\maketitle

    Um vetor $\vt{u}$ pode ser representado através da combinação linear da base cartesiana
    \[\vt{u}=a\base{x}+b\base{y}\]

    Uma rotação de $\theta$ da base cartesiana no círculo trigonométrico unitário é dada por
    \begin{center}
    \begin{tikzpicture}[
        vetor/.style = {->, thick},
        braces/.style = {decorate, decoration = {brace, amplitude=5pt, raise=3}}
    ]
        \pgfmathsetmacro{\l}{3}
        \pgfmathsetmacro{\r}{\l*0.8}
        \pgfmathsetmacro{\angulo}{30}

        \draw (0,0) circle (\r);

        \draw[dashed, thick] ({\r*cos(\angulo)},0) -- (\angulo:\r);
        \draw[dashed, thick] (0,{\r*sin(\angulo)}) -- (\angulo:\r);
        
        \draw[dashed, thick] ({-\r*sin(\angulo)},0) -- (\angulo+90:\r);
        \draw[dashed, thick] (0,{\r*cos(\angulo)}) -- (\angulo+90:\r);
        
        \draw[vetor, blue] (0,0) -- (\angulo:\r);
        \draw[vetor, red] (0,0) -- (\angulo+90:\r);

        \draw[vetor] (\r*0.35,0) arc (0:\angulo:\r*0.35) node[right, yshift=-5, xshift=2]{$\theta$};
        \draw[vetor] (0,\r*0.35) arc (90:90+\angulo:\r*0.35) node[above right, yshift=2]{$\theta$};

        \draw [braces] ({\r*cos(\angulo)},0) -- (0.05,0) node[shift={(0,-15pt)}, pos=0.5, black]{$\cos(\theta)$};

        \draw [braces] ({-\r*sin(\angulo)},0.05) -- ({-\r*sin(\angulo)},{\r*cos(\angulo)}) node[left=7pt, pos=0.5]{$\cos(\theta)$};

        \draw [decorate, decoration = {brace, mirror, raise=3, amplitude=5pt}] ({-\r*sin(\angulo)},0) -- (-0.05,0) node[shift={(-5pt,-15pt)}, pos=0.5]{$-\sin(\theta)$};

        \draw [braces, decoration= {mirror}] ({\r*cos(\angulo)},0.05) -- ({\r*cos(\angulo)},{\r*sin(\angulo)}) node[right=7pt, pos=0.5]{$\sin(\theta)$};
        
        \draw[vetor] (-\l,0)--(\l,0) node[right]{$x$};
        \draw[vetor] (0,-\l)--(0,\l) node[above]{$y$};

    \end{tikzpicture}
\end{center}
    \[
        \rot_\theta(\base{x})=
        \begin{bmatrix}
            \cos(\theta)\\
            \sin(\theta)
        \end{bmatrix}
        \;\text{ e }\;
        \rot_\theta(\base{y})=
        \begin{bmatrix}
            -\sin(\theta)\\
            \cos(\theta)
        \end{bmatrix}
    \]
    
    A rotação no plano preserva algumas propriedades úteis. Com $k$ e $\theta$ $\in\R$, um vetor $\vt{u}$ rotacionado por $\theta$ e depois escalonado por $k$ é o mesmo que um vetor $\vt{u}$ escalonado por $k$ e depois rotacionado por $\theta$.
    \[k\rot_\theta(\vt{u})=\rot_\theta(k\vt{u})\]

    Isso significa que rotacionar um vetor $\vt{u}$ é o mesmo que rotacionar a base cartesiana e escaloná-la pelas componentes de $\vt{u}$.
    \[\rot_\theta(\vt{u})=a\cdot\rot_\theta(\base{x})+b\cdot\rot_\theta(\base{y})\]
    \[
        \rot_\theta(\vt{u})=
        a
        \begin{bmatrix}
            \cos(\theta)\\
            \sin(\theta)
        \end{bmatrix}
        +
        b
        \begin{bmatrix}
            -\sin(\theta)\\
            \cos(\theta)
        \end{bmatrix}
    \]
    \begin{equation}
        \label{equacao-rocatao-vetor}
        \rot_\theta(\vt{u})=
        \begin{bmatrix}
            a\cos(\theta)-b\sin(\theta)\\
            a\sin(\theta)+b\cos(\theta)
        \end{bmatrix}
    \end{equation}
    
    Ao rotacionarmos, por exemplo, $\base{x}$ primeiramente por $\alpha$ e depois por $\beta$, teremos o mesmo resultado se tivéssemos rotacionado por $\alpha+\beta$, comutativamente.
    \[
        \rot_{\alpha+\beta}(\base{x})=\rot_\alpha(\rot_\beta(\base{x}))=\rot_\beta(\rot_\alpha(\base{x}))
    \]
    
    Temos, ainda
    \begin{equation}
        \label{equacao-rotacao-dupla}
        \rot_\beta(\rot_\alpha(\base{x}))
        =
        \rot_\beta\left(
            \begin{bmatrix}
                \cos(\alpha)\\
                \sin(\alpha)
            \end{bmatrix}
        \right)
    \end{equation}
    
    Lembrando a Equação (\ref{equacao-rocatao-vetor}), temos
    \[
        \rot_\theta(\vt{u})
        =
        \begin{bmatrix}
            \color{red}a\color{black}\cos(\theta)-\color{blue}b\color{black}\sin(\theta)\\
            \color{red}a\color{black}\sin(\theta)+\color{blue}b\color{black}\cos(\theta)
        \end{bmatrix}
    \]

    Tomando $\color{red}a=\cos(\alpha)$ e $\color{blue}b=\sin(\alpha)$ nas Equações (\ref{equacao-rocatao-vetor}) e (\ref{equacao-rotacao-dupla}), e considerando a rotação por $\beta$ da Equação (\ref{equacao-rotacao-dupla}), temos
    \[
        \rot_{\alpha+\beta}(\base{x})
        =
        \begin{bmatrix}
            \cos(\alpha+\beta)\\
            \sin(\alpha+\beta)
        \end{bmatrix}
        =
        \begin{bmatrix}
            \color{red}\cos(\alpha)\color{black}\cos(\beta)-\color{blue}\sin(\alpha)\color{black}\sin(\beta)\\
            \color{red}\cos(\alpha)\color{black}\sin(\beta)+\color{blue}\sin(\alpha)\color{black}\cos(\beta)
        \end{bmatrix}
    \]
    
\end{document}
