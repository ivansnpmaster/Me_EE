% Acentuação
\usepackage[utf8]{inputenc}
\usepackage[portuguese]{babel}
\usepackage[T1]{fontenc}

% Configurando as margens
\usepackage[top = 2cm, bottom = 2cm, left = 3cm, right = 3cm]{geometry}
% Indentação do parágrafo
\setlength{\parindent}{2cm}
% Espaçamento entre parágrafo e texto
\setlength{\parskip}{1em}
% Espaçamento entre linhas
\renewcommand{\baselinestretch}{1.5}
% Identar o primeiro parágrafo das seções
\usepackage{indentfirst}
% Para usar símbolos como >=
\usepackage{amssymb}
% Para colocar texto entre $$ com \text{oi}
\usepackage{amsmath}
% Para usar listas com estilos específicos
\usepackage{enumerate}
% Para utilizar imagens
\usepackage{graphicx}
% Posicionamento de imagem
\usepackage{float}
% Para símbolo de permilagem com \textperthousand em modo texto
\usepackage{textcomp}
% Notação de conjuntos: $\mathbb{N, Z, Q, R, C}$
\usepackage{amssymb}

\renewcommand{\figurename}{Figura}
\renewcommand{\tablename}{Tabela}