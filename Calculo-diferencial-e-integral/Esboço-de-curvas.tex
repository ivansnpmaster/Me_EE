\begin{enumerate}[a.]
	
	\item \textbf{Domínio}:
	
	É interessante saber onde a função está definida para sabermos quais intervalos analisar nas etapas seguintes.
	
	\item \textbf{Simetria}:
	
	Quando $f(x)=f(-x)$, a função é par, ou seja, possui simetria em relação ao eixo $y$.
	Quando $f(x)=-f(x)$, a função é ímpar, ou seja, possui simetria rotacionando a função em $180^{\circ}$ em torno da origem.
	É interessante saber se uma função possui simetria, pois o trabalho em descobrir o comportamento da função é menor.
	
	\item \textbf{Assíntotas}:
	
	Para verificar se a função possui assíntotas horizontais, devemos checar como ela se comporta quando tende aos infinitos. Em outras palavras, se $$\lim_{x \rightarrow \pm \infty} f(x) = L$$
	Para verificar se a função possui assíntotas verticais, é interessante verificar pontos próximos do local onde a função não está definida (extremos do domínio). Em outras palavras, se $$\lim_{x \rightarrow a} f(x) = \infty \text{ ou} -\infty$$
	Uma função pode ter uma assíntota vertical quando se aproxima de um número $a$ pela esquerda ou pela direita (ou ambos).
	
	\item \textbf{Intervalos de crescimento/caimento e números críticos}:
	
	Os intervalos de crescimento e caimento de uma função $f(x)$ são obtidos quando $f'(x) = 0$. Quando isolamos a variável independente, encontramos onde a inclinação é $0$ e podemos dividir o domínio de $f$ com base nos nesses pontos. Ao fazer isso, podemos testar pontos contidos nesses intervalos em $f'$ para verificar seu crescimento/caimento com base no sinal de $f'$.
	
	Os pontos onde $f'(x) = 0$ são números críticos $(c)$, indicando possíveis máximos ou mínimos locais.
	
	O ponto $c$ será um máximo local quando $f'(x) \textgreater 0$ em $x \rightarrow c^-$ e quando $f'(x) \textless 0$ em $x \rightarrow c^+$. De forma análoga, $c$ será um mínimo local quando $f'(x) \textless 0$ em $x \rightarrow c^-$ e quando $f'(x) \textgreater 0$ em $x \rightarrow c^+$.
	
	Para sabermos qual número $c_n$ nos possíveis $n$ pontos críticos de $f$ é um máximo ou mínimo absoluto, devemos encontrar $f(c_n)$ e verificar qual número $c$ resultou num maior número $y$ (máximo absoluto) e menor $y$ (menor absoluto).
	
	\item \textbf{Concavidade e pontos de inflexão}:
	
	A concavidade de uma função $f(x)$ é definida pelo sinal de $f''(x)$. Os pontos onde $f''(x) = 0$ são chamados de pontos de inflexão e são neles onde ocorre a mudança de concavidade de $f$. Encontrar a variável independente quando $f''(x) = 0$ divide o domínio de $f$ nos pontos onde ocorre a inflexão da curva.

\end{enumerate}