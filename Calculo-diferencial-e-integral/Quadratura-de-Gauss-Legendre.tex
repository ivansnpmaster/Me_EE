É uma forma de integração numérica. Evalua-se uma área sob uma linha reta pela junção de \textbf{quaisquer} $n$ pontos numa curva ao invés de simplesmente escolher pontos finais. A chave é escolher uma linha que \textbf{balanceia} os erros positivos e negativos.

A Quadratura de Gauss com \textbf{1 ponto}:
\begin{equation}
	\label{equation-gauss-1-ponto-1}
	\int_{a}^b f(x)\;dx \approx \omega_1f(x_1)
\end{equation} 

E assume-se $f(x)\approx c_0+c_1x$. Substituindo-se em ambos os lados da Equação (\ref{equation-gauss-1-ponto-1}), tem-se:
\[ \int_{a}^b (c_0+c_1x)\;dx \approx \omega_1(c_0+c_1x) \]
\[ \left(c_0x+\frac{c_1x^2}{2}\right) \Bigg|_a^b = c_0\omega_1+c_1\omega_1 \]
\[ c_0(b-a)+c_1\frac{b^2-a^2}{2} = c_0\omega_1+c_1\omega_1 \]

Logo, $\omega_1=b-a$ e $\omega_1x_1=\displaystyle\frac{b^2-a^2}{2}$.

Ainda,
\[ \omega_1x_1=\displaystyle\frac{b^2-a^2}{2} \implies (b-a)x_1=\frac{(b-a)(b+a)}{2} \implies x_1=\frac{b+a}{2} \]

Portanto, a aproximação que se deseja encontrar a partir da Equação (\ref{equation-gauss-1-ponto-1}):
\begin{equation}
	\label{equation-gauss-1-ponto-2}
	\int_a^b f(x)\;dx \approx (b-a)f\left(\frac{b+a}{2}\right)
\end{equation}

A aproximação será exata se $f(x)$ for um polinômio de grau $m=2n-1=2\cdot1-1=1$.

Exemplo 1) Encontrar a integral analitica e por Quadratura de Gauss com 1 ponto do polinômio $f(x)=3x+1$.

Quando $x=0 \rightarrow f(0)=1$ e $f(x)=0\rightarrow x=-1/3$, i.e., $a=-1/3$ e $b=0$.

Utilizando a Equação (\ref{equation-gauss-1-ponto-2}):
\[ (b-a)f\left(\frac{b+a}{2}\right)= \left[0-\left(-\frac{1}{3}\right)\right] f\left[\frac{0+\left(-\frac{1}{3}\right)}{2}\right] =\frac{1}{3} f\left(-\frac{1}{6}\right) \]
\[ \frac{1}{3} \left[ 3\left(-\frac{1}{6}\right)+1 \right] = \frac{1}{6} \]

Agora, de forma analítica:
\[ \int_{-1/3}^0 (3x+1)\;dx = \left(\frac{3x^2}{2}+x\right)\Bigg|_{-1/3}^0 = 0-\left[ \frac{3(-1/3)^2}{2}+\left(-\frac{1}{3}\right) \right] =\frac{1}{6} \]

Quando o intervalo a ser integrado for $[-1,1]$ (Gauss-Legendre de 1 ponto), tem-se, a partir de (\ref{equation-gauss-1-ponto-2})::
\[ \int_{-1}^1f(x)\;dx=2f(0) \implies \omega_1=2\;\text{e}\;x_1=0 \]

A Quadratura de Gauss com \textbf{2 pontos}:
\begin{equation}
	\label{equation-gauss-2-pontos-1}
	\int_{a}^b f(x)\;dx \approx \omega_1f(x_1)+\omega_2f(x_2)
\end{equation}

E assume-se $f(x)\approx c_0+c_1x+c_2x^2+c_3x^3$. Substituindo-se em ambos os lados da Equação (\ref{equation-gauss-2-pontos-1}), tem-se:
\begin{equation}
	\label{equation-gauss-2-pontos-2}
	\int_{a}^b (c_0+c_1x+c_2x^2+c_3x^3)\;dx \approx \omega_1(c_0+c_1x_1+c_2x_1^2+c_3x_1^3)+\omega_2(c_0+c_1x_2+c_2x_2^2+c_3x_2^3)
\end{equation}

A parte esquerda de (\ref{equation-gauss-2-pontos-2}):
\[ c_0(b-a)+c_1\frac{(b^2-a^2)}{2}+c_2\frac{(b^3-a^3)}{3}+c_3\frac{(b^4-a^4)}{4} \]

E a parte direita de (\ref{equation-gauss-2-pontos-2}):
\[ c_0(\omega_1+\omega_2)+c_1(\omega_1x_1+\omega_2x_2)+c_2(\omega_1x_1^2+\omega_2x_2^2)+c_3(\omega_1x_1^3+\omega_2x_2^3) \]

Logo,
\[
\begin{cases}
	\omega_1+\omega_2=b-a \\
	\omega_1x_1+\omega_2x_2=\frac{b^2-a^2}{2} \\
	\omega_1x_1^2+\omega_2x_2^2=\frac{b^3-a^3}{3} \\
	\omega_1x_1^3+\omega_2x_2^3=\frac{b^4-a^4}{4}
\end{cases}
\]

As soluções que interessam são $a\leqslant x_1,\;x_2\leqslant b$, pois o sistema pode ter até 3 soluções dado o grau do polinômio.

A solução do sistema acima fornece:
\[ \omega_1=\omega_2=\frac{b-a}{2} \]
\[ x_1=\frac{b-a}{2} \left(-\frac{1}{\sqrt{3}}\right)+\frac{b+a}{2} \]
\[ x_2=\frac{b-a}{2} \left(\frac{1}{\sqrt{3}}\right)+\frac{b+a}{2} \]

Quando o intervalo a ser integrado for $[-1,1]$ (Gauss-Legendre de 2 pontos), tem-se, a partir de (\ref{equation-gauss-2-pontos-1}):
\[ \int_{-1}^1f(x)\;dx\approx1f\left(-\frac{1}{3}\right)+1f\left(\frac{1}{3}\right) \implies \omega_1=\omega_2=1 \;\text{e}\; x_1=-x_2=-\frac{1}{3} \]